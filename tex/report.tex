\documentclass[a4paper,12pt]{article}
\usepackage{graphicx}
\usepackage{authblk}
\usepackage{hyperref}

\title{Definition Prediction Based on Etymology}
\author[1]{Noah Gardner}
\date{$9/24/2021$}
\affil[1]{College of Computing and Software Engineering, Kennesaw State University}

\begin{document}
\maketitle

\section{Abstract}
There are many different methods to extract the defition of a word given
information about it besides the definition. Sometimes, the definition can be
extracted from the context of the word, but the context is not always available.
In every case, however, the word itself is available, and we if we can find the
etymology used to build the word, we have a good hint of the definition of the
word. Automically generated definitions could lower the cost of crowd-sourced
annotations. This research proposes a method to extract the definition of a word
by parsing the etymology of the word.

\section{Introduction}
Wiktionary\footnote{wiktionary.org} is an online dictionary that provides
details about many words, including etymology, definition, pronounciation, and
examples. Wiktionary also supports many different languages. Some research has
used data dumped from Wikitionary to support NLP research such as etymology
prediction and word sense disambiguation.

The etymology of a word is a tree structure that describes the word's origin.
The tree is a series of nodes, each of which has a parent and a child. The root
node is the word itself. The child nodes are the words that are used to build
the word.

\section{Literature Review}
\begin{itemize}
    \item[1] Computational Etymology and Word Emergence
        \cite{wu_computational_2020}

        Wu et al. describe a comprehensive Wikitionary parser that allows them
        to predict the etymology of a word across multiple languages. They use a
        LSTM model for three different experimental settings that explore the
        relationship between words and their etymology. They also show their
        model is capabable of predicting the parent language of a word given
        it's relationship.

    \item[2] Augmenting semantic lexicons using word embeddings and transfer
        learning \cite{alshaabi_augmenting_2021}

        Al-Shaabi et al. describe a method to augment the semantic lexicons of a
        language using word embeddings and transfer learning. They argue that
        lexicon-based models for sentiment analysis systems are more
        interpretable and easier to use than contextual models. However, it can
        be challenging to add sentiment information to new words in the lexicon.
        They propose two models that are to predict sentiment scores using word
        embeddings and transfer learning. Their methods show human-level
        performance on a dataset of annotated sentiment scores.

    \item[3] Deep contextualized word representations \cite{peters_deep_2018}

        Peters et al. describe a deep representation model for words that can be
        used across NLP tasks such as sentiment analysis and textual entailment.
        They argue that high quality representations are difficult to learn, and
        should ideally model characteristics of the word and how it's uses can
        vary in different contexts. They propose a model that uses both forward
        and backward contextualized word embeddings. Their methods obtain
        benchmark results on main NLP tasks and can be incorporated into many
        modern NLP systems.
\end{itemize}

\section{Methodology}
\subsection{Dataset}
The dataset used for this project will come from \textit{Yawipa Data
    Extract}\footnote{cs.jhu.edu/~winston/yawipa-data.html} dataset that parsed
information on each word from Wiktionary, including definitions and
etymology \cite{wu_computational_2020}.

\subsection{Model}
Given a supportive dataset and modern NLP algorithms, it may be possible to
train a model to predict the definition of a word given the etymology of the
word. The definitons will also come from the Wikitionary dataset but are also
available from different datasets and API's.

If it is impossible or proves too difficult to generate the definition of a
word, then at least this project will attempt to provide statistical
correlations between etymology and definitions.

\bibliographystyle{unsrt}
\bibliography{report}

\end{document}
