\documentclass[11pt,a4paper]{article}
\usepackage[hyperref]{acl2020}
\usepackage{times}
\usepackage{latexsym}
\renewcommand{\UrlFont}{\ttfamily\small}
\usepackage{microtype}
\newcommand\BibTeX{B\textsc{ib}\TeX}
\usepackage{xcolor}

\aclfinalcopy % Uncomment this line for the final submission
\title{Etymological Embeddings for Contexless Definition Modeling}
\author{Noah Gardner \\
  Kennesaw State University \\
  College of Computing and Software Engineering \\
  \texttt{ngardn10@students.kennesaw.edu} \\}

\begin{document}
\maketitle
\begin{abstract}
  Definition modeling is the problem of estimating the probability of an output
  definition given an input word embedding. There exist some methods of creating
  word embeddings concatenated that take the input word concatenated with the
  context of the word. However, the context is not always available.
  Additionally, progress has been made in research for \textit{etymology
    modeling} where the etymology of a word can be estimated from the input word
  embedding. In this paper, we propose a definition modeling method that uses
  etymological information.
\end{abstract}

\section{Introduction}
Word embeddings are vector representations of words that allow us to use words
as inputs to machine learning models for natural language processing tasks
\cite{mikolov_efficient_2013}. There are many word embedding methods that
achieve state-of-the-art performance on NLP problems, such as sentiment
analysis. Addtionally, contextualized word embeddings have been shown to improve
performance with models such as ELMo and BERT \cite{peters_deep_2018,
  devlin_bert_2019}.

Dictionary definitions can yield value for sentiment aware models, however,
crowdsourced annoations are costly. The task of definition modeling was proposed
to address this problem. The goal of definition modeling is to estimate the
probability of an output definition given an input word embedding
\cite{noraset_definition_2016}.

Wiktionary\footnote{wiktionary.org} is a free online dictionary that provides
details about many words, including definitions, etymologies, pronounciation,
and examples. Some research has used the large data dumped from Wikitionary to
support NLP research such as etymology modeling and word sense disambiguation
\cite{wu_computational_2020}.

The etymology of a word is a tree structure that describes the word's origin.
Although contextualized embeddings show improvement in NLP tasks, the context of
a word is not always available. With advances in etymology modeling, if we know
the source language of a word, we can predict the etymology of the word. Using
this observation, we propose a word embedding with etymological information for
the task of definition modeling. This work intends to show improvement for
contextless definition modeling, although it may be used in conjuction with a
contextualized embedding for even better performance.

\section{Related Work}

Definition modeling was intially described by \citet{noraset_definition_2016}.
Their research is based on a recurrent neural network language model
\cite{mikolov_recurrent_2010} with a modified recurrent unit. They use the word
to be defined placed at the beginning of the definition so the model will see
the word only on the first step.

\citet{chang_what_2019} explore contextualized embedding for definition
modeling. They reformulate the problem of definition modeling from text
generation to text classification. Their results show state-of-the-art
performance on the task of definition modeling.

\citet{washio_bridging_2019} proposed a method for context-based definition
modeling that considers the semantic relations between both the word to be
defined and the words in the definition. They apply semantic information to both
the definition encoder and decoder.

\citet{barba_exemplification_2021} introduce exemplification modeling, an
adjacent problem to definition modeling that uses a definition embedding to
generate possible example sentences. They use a sequence-to-sequence based
approach and show near human-level annotation performance. Their problem is
similar in that they use the definition as context to create example sentences.

\section{Overview}
In this section, we investigate the parsed wikitionary dump from
\citet{wu_computational_2020} and discuss the relevant aspects of the
dataset.

\subsection{Definition}

\begin{figure}[h]
  \begin{tabular}{|l|l|}
    \hline
    \textbf{Word} & \textbf{Definition}             \\
    \hline\hline
    free          & $(lb|en|social)$ Unconstrained. \\
    free          & Not imprisoned or enslaved.     \\
    free          & Generous; liberal.              \\
    \hline
  \end{tabular}
  \caption{Parsed wikitionary example definitions for the english word
    \textit{free}.}
  \label{fig:wiktionary_definitions}
\end{figure}

The definition dataset includes information on the source language, the word to
be defined, the part of speech, and the defintion of the word. The definition of
a word can also contain a specific context in which the definition is used.
Figure \ref{fig:wiktionary_definitions} shows some example definitions from the
dataset for the word \textit{free}.

\subsection{Etymology}

Similar to the definition dataset, the etymology dataset includes information on
the source language, the word to be analyzed, and the etymology of the word. The
etymology is a tree structure that describes the word's origin, including roots
from other languages. Figure \ref{fig:wiktionary_etymology} shows an example
etymology from the dataset for the word \textit{free}.

\begin{figure}[h]
  \color{blue}free \color{black}[\color{green}(eng|free)\color{blue}(root)\color{red}(eng|ine-pro|*preyH-)\color{black}]
  \caption{Parsed wikitionary example etymology for the english word
    \textit{free}.}
  \label{fig:wiktionary_etymology}
\end{figure}

\section{Methodology}

From the definition and etymology datasets, we use only the words with the
english source language. For the defintions, we remove self-referential
defintions and utilize a maximum of three definitions per word. Although the
context-based definitions may provide some benefit for a context-based model, we
remove the context from the definitions. Additionally, the definition of some
words are completely context (such as alternate spellings) and are also removed.
For the etymologies, we use only the first etymology for each word if there
exist multiple. Finally, we ignore words tagged with \textit{proper noun}.
Dataset statistics are shown in Figure \ref{fig:dataset_stats} after the
described steps are applied.

\begin{figure}[h]
  \begin{tabular}{|l|l|}
    \hline
    \textbf{Type}                & \textbf{Amount} \\
    \hline\hline
    Unique Words                 & \textbf{320855} \\
    Average Definitions Per Word & \textbf{1.288}  \\
    Etymology (Average Length)   & \textbf{41.725} \\
    Definition (Average Length)  & \textbf{51.711} \\
    \hline
  \end{tabular}
  \caption{Dataset statistics for the combined datasets.}
  \label{fig:dataset_stats}
\end{figure}

\section{Experimental Results}

\section{Conclusion}

\section*{Acknowledgments}
This work was supported by computational resources provided by the Kennesaw
State University Department of Electrical and Computer Engineering.

\bibliography{report.bib}
\bibliographystyle{acl_natbib}
\end{document}
